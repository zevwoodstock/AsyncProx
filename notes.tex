\documentclass[a4paper,11pt,fleqn]{article}
%\usepackage{showkeys}
%\usepackage{refcheck}%check unused refs if showkeys activated
\usepackage{cite}
\usepackage{amsmath,mathtools}
\usepackage{amssymb}
%-----------
%\usepackage[usenames,dvipsnames]{pstricks}
%\usepackage{pstricks-add}
%\usepackage{pst-grad}
%\usepackage{pst-plot}
%\usepackage{pst-eucl}
%\psset{algebraic}
%--------
\usepackage{tikz}
\usepackage{pgfplots}
\usepackage{makeidx}
%\pgfplotsset{compat=1.15}
\usepackage{mathrsfs}
\usetikzlibrary{arrows}
\definecolor{ffwwqq}{rgb}{1.,0.4,0.}
\definecolor{qqzzqq}{rgb}{0.,0.6,0.}
\definecolor{qqqqff}{rgb}{0.,0.,1.}
\definecolor{dred}{HTML}{D90404}
\definecolor{orng}{HTML}{D35400}
\newcommand{\plc}[1]{{\color{dred}\sffamily #1}}
\newcommand{\zcw}[1]{{\color{orng}\sffamily #1}}
\newcommand{\dft}{\ensuremath{\operatorname{DFT}}\,}

\usepackage{amssymb}
\usepackage{mathrsfs} %pour mathscr
\usepackage{theorem} 
\usepackage{booktabs}
\usepackage{euscript}
\usepackage{exscale,relsize}
\usepackage{graphicx}
\usepackage{booktabs}
\usepackage{srcltx}
\usepackage{xcolor}
\usepackage{charter}
\usepackage{pifont}
%\usepackage{mathabx}
%\usepackage{ntheorem}
%\usepackage[T1]{fontenc}
%\usepackage[bitstream-charter]{mathdesign}
\usepackage{url}

\newcommand{\email}[1]{\href{mailto:#1}{\nolinkurl{#1}}}
%------------------------------------------------------------------
\topmargin-0.0cm
\oddsidemargin -0.0cm
\textwidth16.5cm 
\headheight0.0cm
\textheight21.3cm
\parindent6mm
\parskip9pt
\tolerance1000
%\usepackage{amsthm}
\usepackage{caption}
\usepackage{subcaption}
\usepackage{amsmath,color}
\setlength\parindent{15pt} % for paragraph alignment
\usepackage{wasysym}
\usepackage{amssymb}
\usepackage{epstopdf}
\usepackage{float}
\usepackage{mathrsfs}
%\usepackage{subfig}
\usetikzlibrary{arrows}
\definecolor{xdxdff}{rgb}{0.6588235294117647,
0.6588235294117647,0.6588}
\definecolor{qqqqff}{rgb}{0.3333333333333333,
0.3333333333333333,0.3333}
\DeclareCaptionFormat{cont}{#1 (cont.)#2#3\par}

%\usepackage[usenames,dvipsnames]{xcolor}
%\date{May 11, 2017}
%\newtheorem{theorem}{Theorem}[section]
%\newtheorem{corollary}[theorem]{Corollary}
%\newtheorem{proposition}[theorem]{Proposition}
%\newtheorem{lemma}[theorem]{Lemma}
%\newtheorem{example}[theorem]{Example}
%\newtheorem{notation}[theorem]{Notation}
%\newtheorem{definition}[theorem]{Definition}
%\newtheorem{condition}[theorem]{Condition}
%\newtheorem{algorithm}[theorem]{Algorithm}
%\newtheorem{remark}[theorem]{Remark}
%\newtheorem{problem}[theorem]{Problem}
%\newtheorem{conjecture}[theorem]{Conjecture}
%\newtheorem{question}[theorem]{Question}
\newcommand{\vertiii}[1]{{\left\vert\kern-0.25ex\left\vert
\kern-0.25ex\left\vert #1 
\right\vert\kern-0.25ex\right\vert\kern-0.25ex
\right\vert}}
\newtheorem{theorem}{Theorem}[section]
\newtheorem{lemma}[theorem]{Lemma}
\newtheorem{corollary}[theorem]{Corollary}
\newtheorem{question}[theorem]{Question}
\newtheorem{proposition}[theorem]{Proposition}
\newtheorem{assumption}[theorem]{Assumption}
\theoremstyle{plain}{\theorembodyfont{\rmfamily}%
\newtheorem{conjecture}[theorem]{Conjecture}}
\theoremstyle{plain}{\theorembodyfont{\rmfamily}%
\newtheorem{example}[theorem]{Example}}
\theoremstyle{plain}{\theorembodyfont{\rmfamily}%
\newtheorem{remark}[theorem]{Remark}}
\theoremstyle{plain}{\theorembodyfont{\rmfamily}%
\newtheorem{algorithm}[theorem]{Algorithm}}
\theoremstyle{plain}{\theorembodyfont{\rmfamily}%
\newtheorem{condition}[theorem]{Condition}}
\theoremstyle{plain}{\theorembodyfont{\rmfamily}%
\newtheorem{definition}[theorem]{Definition}}
\theoremstyle{plain}{\theorembodyfont{\rmfamily}
\newtheorem{fact}[theorem]{Fact}}
\theoremstyle{plain}{\theorembodyfont{\rmfamily}
\newtheorem{problem}[theorem]{Problem}}
\theoremstyle{plain}{\theorembodyfont{\rmfamily}
\newtheorem{notation}[theorem]{Notation}}
\theoremstyle{plain}{\theorembodyfont{\rmfamily}
\newtheorem{project}[theorem]{Project}}
\def\proof{{\it Proof}. \ignorespaces}
\def\endproof{\vbox{\hrule height0.6pt\hbox{\vrule height1.3ex% 
width0.6pt\hskip0.8ex\vrule width0.6pt}\hrule
height0.6pt}}
\renewcommand\theenumi{(\roman{enumi})}
\renewcommand\theenumii{(\alph{enumii})}
\renewcommand{\labelenumi}{\rm (\roman{enumi})}
\renewcommand{\labelenumii}{\rm (\alph{enumii})}
\newcommand{\cone}{\ensuremath{\text{\rm cone}\,}}
\newcommand{\spts}{\ensuremath{\text{\rm{spts}\,}}}
\newcommand{\Fix}{\ensuremath{\text{\rm Fix}\,}}
\newcommand{\sign}{\ensuremath{\text{\rm sign}\,}}
\newcommand{\spa}{\ensuremath{\text{span}\,}}
\newcommand{\soft}[1]{\ensuremath{{\operatorname{soft}}_{{#1}}\,}}
\newcommand{\hard}[1]{\ensuremath{{\operatorname{hard}}_{{#1}}\,}}
\newcommand{\Argmin}{\ensuremath{{\text{\rm Argmin}}}}
\newcommand{\Scal}[2]{\bigg\langle{#1}\;\bigg|\:{#2}\bigg\rangle}
\newcommand{\scal}[2]{{\langle{{#1}\mid{#2}}\rangle}}
\newcommand{\sscal}[2]{{\big\langle{{#1}\mid{#2}}\big\rangle}}
\newcommand{\abscal}[2]{\left|\left\langle{{#1}\mid{#2}}}%
\newcommand{\emp}{\ensuremath{{\varnothing}}}
\newcommand{\minimize}[2]{\ensuremath{\underset{\substack{{#1}}}%
{\text{minimize}}\;\;#2 }}
%\numberwithin{equation}{section}
\setlength{\itemsep}{1pt} 


%-------------------------------- HYPERLINKS -------------------
%http://tug.ctan.org/cgi-bin/ctanPackageInformation.py?id=hyperref
\definecolor{labelkey}{rgb}{0,0.08,0.45}
\definecolor{refkey}{rgb}{0,0.6,0.0}
\definecolor{Brown}{rgb}{0.45,0.0,0.05}
\definecolor{dgreen}{rgb}{0.00,0.49,0.00}
\definecolor{dblue}{rgb}{0,0.08,0.75}
\RequirePackage[,colorlinks,hyperindex]{hyperref} %omitted dvips here
\hypersetup{linktocpage=true,citecolor=dblue,linkcolor=dgreen}
%--------------------------------------------------------------------
%Command \theoremstyle already defined. }
%Package natbib Error: Bibliography not compatible with author-year 
%citations. ...and\NAT@force@numbers{}\NAT@force@numbers
\usepackage{natbib}
\setlength{\bibsep}{3.5pt}
\def\HH{\mathcal H}
\def\BHG{\mathcal {B}(\mathcal{H},\mathcal{G})}
\def\card{\textnormal{card}}
\def\GG{\mathcal G}
\def\KK{\mathcal K}
\def\G0{\Gamma_0(\mathcal G)}
\def\H0{\Gamma_0(\mathcal H)}
\newcommand{\NN}{\ensuremath{\mathbb N}}
\newcommand{\RR}{\ensuremath{\mathbb{R}}}
\newcommand{\iin}{\ensuremath{\textrm{i}(n)}}
\newcommand{\ii}{\ensuremath{\textrm{i}}}
\newcommand{\RP}{\ensuremath{\left[0,+\infty\right[}}
\newcommand{\RM}{\ensuremath{\left]-\infty,0\right]}}
\newcommand{\RMM}{\ensuremath{\left]-\infty,0\right[}}
\newcommand{\BL}{\ensuremath{\EuScript B}\,}
\newcommand{\RPP}{\ensuremath{\left]0,+\infty\right[}}
\newcommand{\RPPX}{\ensuremath{\left]0,+\infty\right]}}
\newcommand{\ran}{\ensuremath{\text{\rm ran}\,}}
%\newcommand{\span}{\ensuremath{\text{\rm span}\,}}
\def\sri{\textnormal{sri}}
\newcommand{\reli}{\ensuremath{\text{\rm ri}\,}}
\newcommand{\pushfwd}{\ensuremath{\mbox{\Large$\,\triangleright\,$}
}}
%\def\Id{\text{Id}}
\newcommand{\menge}[2]{\big\{{#1}~\big |~{#2}\big\}}
\newcommand{\Menge}[2]{\left\{{#1}~\left|~{#2}\right.\right\}}

\newcommand*{\Id}{\text{\normalfont Id}}
\usepackage{lipsum}
\setcitestyle{square,citesep={,}}
\newcommand\scalemath[2]{\scalebox{#1}{\mbox{\ensuremath
{\displaystyle #2}}}}
\newcommand{\dom}{\ensuremath{\text{\rm dom}\,}}
\newcommand{\ri}{\ensuremath{\text{\rm ri}\,}}
\newcommand{\bdry}{\ensuremath{\text{\rm bdry}\,}}
\newcommand{\Int}{\ensuremath{\text{\rm int}\,}}
\newcommand{\supp}{\ensuremath{\text{\rm supp}\,}}
\newcommand{\gra}{\ensuremath{\text{\rm gra}\,}}
\newcommand{\prox}{\ensuremath{\text{\rm Prox}\,}}
\newcommand{\zer}{\ensuremath{\text{\rm zer}\,}}
\newcommand{\rec}{\ensuremath{\text{\rm rec}\,}}
\newcommand{\core}{\ensuremath{\text{\rm core}}}
\newcommand{\range}{\ensuremath{\text{\rm range}\,}}
\usepackage{lineno,hyperref}
\modulolinenumbers[5]

\bibliographystyle{elsarticle-num}

\begin{document}
%\title{``Whiteboard''}
%\date{}
%\maketitle

Literature:
\begin{itemize}
\item
Glaudin/PLC - Asynchronous implementation has not been studied
\item
Eckstein/PLC - Proven to work for asynchronous, but noone has
programmed it. It is the first-in-class for fully nonsmooth
problems.
\item
Application areas:
low-rank + sparse decomposition -- smooth, although there are
likely problem-specific algos to compare.
\end{itemize}

Preliminary To-Dos:
\begin{itemize}
\item
Keep a daily log of the research you work on -- tasks completed,
progress made, etc. When you learn to write in latex, you will
submit this report compiled in latex. For starting out, .txt is
fine.
\item
Acquire a method to turn .tex files to PDF; (e.g., Overleaf,
pdfLaTeX, TeXShop,
TeXStudio,  LaTeX $\to$ dvips $\to$ ps2pdf, etc. If you use vim, I
can recommend vim-latex.)
\item
Install Julia; familiarize yourself with Julia by running a basic
experiment with the theoretical tools below (e.g., implement a few
basic proximity operators from here
\url{http://proximity-operator.net/indicatorfunctions.html} and
compare with the premade MATLAB/Octave scripts available in their
repo)
\item
\textbf{Theory and Vocabulary Questions} (if a question is not provided,
please define the word \textbf{and} provide an intuitive
description/example.):
\begin{itemize}
\item
What is the definition of a convex subset of $\mathbb{R}$? What
about a convex subset of $\mathbb{R}^n$?
\item
What does it mean for a function $f\colon\mathbb{R}^n\to\mathbb{R}$
to be convex?
\item
{\em Proximity operator} (of a convex function): 
(hint: \url{http://proximity-operator.net/proximityoperator.html})
\item
{\em Projection operator} (of a convex set):
(hint:
\url{https://en.wikipedia.org/wiki/Hilbert_projection_theorem})
\item
An {\em indicator function} of a closed convex set
$C\subset\mathbb{R}^n$ is given by
\begin{equation}
\iota_C\colon\RR^n\to\RR\colon x\mapsto
\begin{cases}
0\;\;\text{if}\;\;x\in C\\
+\infty\;\;\text{otherwise.}
\end{cases}
\end{equation}
Prove that the proximity operator of $\iota_C$ is the projection
operator of $C$.
\item
Define the {\em Gradient} of a convex function
$f\colon\RR^n\to\RR$.
\item
Prove or disprove (via counterexample): The gradient of a convex
function $f\colon\RR^n\to\RR$ always exists.
\item
What is the difference between a gradient and a proximity operator?
\item
What ``direction'' does the gradient point towards?
\item
Provide a geometric description of where the vector $x - \nabla
f(x)$ points.
\end{itemize}
One of the main things we are going to be comparing is various
algorithms which combine gradients, proximity operators, and
projection operators. A special case of these algorithms is
optimization algorithms, which are at the backbone of machine
learning. We are specifically going to research
\textbf{asynchronous}
algorithms involving these tools -- we are going to implement them
on challenging problems, determine the best hyperparameters for
running them, and compare the algorithms between each other.

Many of the papers provided in the Literature folder are involve
{\em firmly nonexpansive operators} -- this is a class which
includes proximity operators, projections, and rescaled gradient
operators (we must rescale by $1/L$ where $L$ is known as the
``Lipschitz constant'' of the gradient. Don't worry too much about
where $L$ comes from for now).

\textbf{Computational goals:}
\begin{itemize}
\item
Implement the main algorithms in Glau20 and Comb18 with the ability
to activate the operators $(T_i)_{i\in I}$ asynchronously.
\end{itemize}
\end{itemize}
\end{document}

